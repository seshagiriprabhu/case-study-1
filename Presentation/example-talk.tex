% This text is Free and open Open Source.
% It's a part of presentation made by myself.
% It may be used only for academic purpose
% May, 2012
% Author: Seshagiri Prabhu
% Amrita Vishwa Vidyapeethm 
% seshagiriprabhu@gmail.com
% www.seshagiriprabhu.wordpress.com

\documentclass[12pt]{beamer}
\usetheme{Oxygen}
\usepackage{thumbpdf}
\usepackage{wasysym}
\usepackage{ucs}
\usepackage[utf8]{inputenc}
\usepackage{pgf,pgfarrows,pgfnodes,pgfautomata,pgfheaps,pgfshade}
\usepackage{verbatim}
\usepackage{listings}
\usepackage{courier}
\usepackage{caption}
\usepackage{verbatim} 
\usepackage{upquote}
\usepackage{graphics}
\usepackage{latexsym}
\usepackage{fixltx2e}
\usepackage{hyperref}
\usepackage{amssymb}
\usepackage{amsmath}
\usepackage{mathtools}
\usepackage{pgf}
\usepackage{fmtcount}% http://ctan.org/pkg/fmtcount
\usepackage{algorithm, algpseudocode}
\usepackage{caption}
\captionsetup[algorithm]{font=scriptsize}
\usepackage{lipsum}
\newcommand\Fontvi{\fontsize{5}{6}\selectfont}
%\renewcommand\tinyv{\@setfontsize\tinyv{4pt}{6}}
%\renewcommand\tiny{\@setfontsize\tiny{4pt}{6}}

\usepackage{xcolor}
\def\SPSB#1#2{\rlap{\textsuperscript{\textcolor{red}{#1}}}\SB{#2}}
\def\SP#1{\textsuperscript{\textcolor{red}{#1}}}
\def\SB#1{\textsubscript{\textcolor{blue}{#1}}}

\pdfinfo
{
  /Title       (Computing the Modular Inverse of a Polynomial Function over GF(2^{P}) Using Bit Wise Operation)
  /Creator     (Seshagiri Prabhu N)
  /Author      (TeX)
}
\title{Computing the Modular Inverse of a Polynomial Function over GF($2^\textsuperscript{P}$)\\ Using Bit Wise Operation}
%\subtitle{Basic Introduction}
\author{Seshagiri Prabhu N 	\\ Guide: Raphael Fourquet}
% \institute[Amrita Vishwa Vidyapeetham] % (optional)
%{
%  \begin{center}
%    \tiny M.Tech, Cyber Security and Networks (First Year)\\
%  	Amrita School of Engineering,
%	Amritapuri Campus
%  \end{center}  
%}

\begin{document}
\frame{\titlepage}
\section*{}
\begin{frame}
  \frametitle{Outline}
  \tableofcontents[section=1,hidesubsections]
\end{frame}

\newcommand{\icon}[1]{\pgfimage[height=1em]{#1}}

%%%%%%%%%%%%%%%%%%%%%%%%%%%%%%%%%%%%%%%%%
%%%%%%%%%% Content starts here %%%%%%%%%%
%%%%%%%%%%%%%%%%%%%%%%%%%%%%%%%%%%%%%%%%%

\section{Introduction}

\begin{frame}{timebomb}
  \frametitle{Introduction}
	\begin{enumerate}
		\item Relevance of Modulo arithmetic in public key crypto system
		\item The use of \textbf{E}xtended \textbf{E}uclidean \textbf{A}lgorithm (\textbf{EEA}) to evaluate the multiplicative inverse
	\end{enumerate}
\end{frame}


\section{Contribution of this paper}
\begin{frame}
\frametitle{Contribution of this paper}
	\begin{block}{Contribution of this paper}
		Computerized algorithm for the determination of the multiplicative inverse of a polynomial over GF(2\SP{P}) using simple bit wise shift and XOR operations. 	
	\end{block}
\end{frame}

\section{Problem Description}
\begin{frame}
\frametitle{Problem Description}
	\begin{block}{EEA}
		Let \textcolor{red}{$A(x)$} and \textcolor{red}{$B(x)$} be polynomials.\\
		EEA gives \textcolor{red}{$U$} and \textcolor{red}{$V$} such that\\
		$\gcd{(A, B)} = U*A + V*B$
	\end{block}
	\begin{block}{Note}
	If \textcolor{red}{$A$} is irreducible, then its $\gcd$ is \textcolor{red}{$1$}, and we are only
	interested in \textcolor{red}{$V$}, which is the inverse of \textcolor{red}{$B [mod A]$}
	\end{block}
\end{frame}
\begin{frame}
\frametitle{Problem Description}
	\begin{block}{Polynomial representation}
	  	The finite field is a representative of a polynomial function with respect to one variable x: \\
	  	GF(2\SP{p}) = x\SP{p-1} + x\SP{p-2} + \dots + x\SP{2} + x\SP{1}
	\end{block}
	\begin{block}{Example}
		Finite field GF(2\SP{8}) = x\SP{8} + x\SP{4} + x\SP{3} + x + 1 \\
		53\SB{10} $\rightarrow$ $1010011$\SB{2} $\rightarrow$ (x\SP{6} + x\SP{4} + x + 1) \\
		The EEA of 53 on GF(2\SP{8}) is x\SP{7} + x\SP{6} + x\SP{3} + x	 
	 \end{block}
\end{frame}

\section{Proposed Algorithm}
\begin{frame}[fragile]
	\frametitle{Proposed Algorithm}
	\begin{algorithm}[H]
		%\algsetup{linenosize=\tiny}
	  	\tiny
		\label{euclid}
		\begin{algorithmic}
			\Procedure{Multiplicative Inverse}{$A_{3}[\hspace{1mm}]$, $B_{3}[\hspace{1mm}]$}
				\State $C\SB{1} = A\SB{2} = B\SB{2} = 0$;
				\While {($B\SB{3} \textgreater 1$)} \Comment{Step 1 do}
					\State $Q = 0$;
					\State $Temp = B\SB{3}$;
					\While {($A\SB{}3 > Temp \parallel BitSize(C) \geq BitSize(Temp)$)}
						\State $Q\SB{1} = 1$;
						\While {($A\SB{3MSB} == B\SB{3MSB}$)}
							\State $B\SB{3} = B\SB{3} << LinearLeftShift$ ;
							\State $Q\SB{1} = Q\SB{1} * 2$;
						\EndWhile
							\State $Q = Q + Q\SB{1}$;
							\State $A\SB{3} = A\SB{3}[\hspace{1mm}] \oplus B\SB{3}[\hspace{1mm}]$;
							\State $B\SB{3} = Temp$;
					\EndWhile
					\State $A\SB{2} = B\SB{2}; B\SB{3} = A\SB{3}; A\SB{3} = Temp$;
					\State $N = BitSize(Q)$; \Comment {Binary Bit Size of Q}
					\State $Temp = B\SB{2}; C\SB{2}=0;$\Comment {Step2}	
					\While{($N\geq1$)}
						\State $C\SB{2} = 0\SB{d}$;
						\If {($Q\SB{N}==1$)}\Comment {Testing if Nth bit of Q is 1}
							\State $C\SB{1}=B\SB{2}<< N-1$; \Comment{Linear left shift by N-1 times}
							\State $C\SB{2} = C\SB{2}\oplus C\SB{1}$;
						\EndIf
						\State $N--$;
					\EndWhile
					\State $B\SB{2} = C\SB{2}; A\SB{2} = Temp; B\SB{2}=B\SB{2} \oplus A\SB{2};$ \Comment  {Multiplicative Inverse}
				\EndWhile
			\EndProcedure
		\end{algorithmic}
%		\caption{\tiny {Euclidean Algorithm}}
	\end{algorithm}
\end{frame}

\section{Implementation of Algorithm}
\begin{frame}
	\frametitle{Implementation of Algorithm}
	\framesubtitle{Let's apply EEA to A = 283 and B = 42}
	\tiny 
	\begin{center}
		\begin{tabular}{| l | c | c | r | r |}
			\hline
			i & Operation & Binary & U & V \\ \hline
			0 & $A$ & $100011011$ & 1 & 0 \\ \hline 
			1 & $B$ & $000101010$ & 0 & 1 \\ \hline
			  & $3<<B$ & $101010000$ & 0 & 1000 \\ 
			2 & $A \leftarrow A \oplus (3<<B)$ & $001001011$ & 1 & 1000 \\ \hline
			 & $1<<B$ & $001010100$ & 0 & 0010 \\
			3 & $A \leftarrow A \oplus (1<<B)$ & $000011111$ & 1 & 1010 \\ \hline
			 & $A<B \hspace{1mm} A \rightleftharpoons B$ & & & \\
			 & $A$ & $000101010$ & 00 & 00001 \\
			 & $B$ & $000011111$ & 01 & 01010 \\
			 & $1<<B$ & $000111110$ & 10 & 10100 \\
			4 & $A \leftarrow A \oplus (1<<B)$ & $000010100$ & 10 & 10101 \\ \hline
			 & $A<B \hspace{1mm} A \rightleftharpoons B$ & & & \\
			 & $A$ & $000011111$ & 01 & 01010 \\
			 & $B$ & $000010100$ & 10 & 10101 \\
			5 & $A \leftarrow A \oplus B$ & $000001011$ & 11 & 11111 \\ \hline
			 & $A<B \hspace{1mm} A \rightleftharpoons B$ & & & \\
			 & $A$ & $000010100$ & 010 & 010101 \\
			 & $B$ & $000001011$ & 011 & 011111 \\
			 & $1<<B$ & $000010110$ & 110 & 111110 \\
			6 & $A \leftarrow A \oplus (1<<B)$ & $000000010$ & 100 &  101011 \\ \hline	
			 & $A<B \hspace{1mm} A \rightleftharpoons B$ & & & \\
			 & $A$ & $000001011$ & 00011 & 00011111 \\
			 & $B$ & $000000010$ & 00100 & 00101011 \\
			 & $2<<B)$ & $000001000$ & 10000 & 10101100 \\
			7 & $A \leftarrow A \oplus (1<<B)$ & $000000011$ & 10011 & 10110011 \\ \hline
			8 & $A \leftarrow A \oplus B$ & $000000001$ & 10111 & 10011000 \\ \hline		 
		\end{tabular}
	\end{center}
\end{frame}

\section{Conclusion}
\begin{frame}
\frametitle{Conclusion}
	\begin{enumerate}
		   \item This algorithm can be easily extended for determining the elements of the S-Box used in $AES$.
	   \item This algorithm is efficient for determining the multiplicative inverse of polynomial over $GF(2\SP{P})$
	\end{enumerate}
\end{frame}

\begin{frame}
\frametitle{Future works}
  \begin{block}{Possible future works}
  	\begin{enumerate}
  		\item Optimize the algorithm
  		\item Comparative study with many existing algorithm
  		\item Implementation in hardware for real time applications
  	\end{enumerate}
  \end{block}
\end{frame}


\frame{
  \vspace{2cm}
  {\huge Questions ?}
  \vspace{3cm}
  \begin{flushright}
	Seshagiri Prabhu

    \structure{\footnotesize{seshagiriprabhu@gmail.com}}
  \end{flushright}
}

\end{document}
